\section{Motivation}
We motivate the need for data enrichment through a real-world example gathered from a large Fortune 500 company. We note here that the described system is currently being testing with the client. 

\textbf{Creating comprehensive customer models}:
		Creating comprehensive customer models has become a holy grail in the consumer business, especially in verticals like retail and healthcare. The information that is collected impacts key decisions
		like inventory management, promotions and rewards, and for delivering a more personal experience. Often businesses engage customers to register for programs like reward cards or connect with the 
		customers using social media. This limited initial engagement gives them the access to basic information about a customer, such as name, email, address, and social media handles. However, in  a vast 
		majority of the cases, such information is incomplete and the gaps are not uniform. For example, for a customer John Doe, a business might have the name, street address, and a phone number, whereas
		for Jane Doe, the available information will be name, email, and a Twitter handle. Leveraging the basic information and completing the gaps, also called as creating a 360 degree customer view has many 
		applications, including:  
		\begin{enumerate}
			\item \textbf{Personalized and targeted promotions}: The more information a business has about its customer, the better it can personalize deals and promotions. Further, business could also 
				use aggregated information (such as interests of people living in a neighborhood) to manage inventory and run local deals and promotions. 
			\item \textbf{Better segmentation and analytics}: The provider may need more information about their customers, beyond what they have in the profile, for better segmentation and analytics. 
				For example, a certain e-commerce site may know a person’s browsing and buying history on their site and have some limited information about the person such as their address and credit card 
				information. However, understanding their professional activities and hobbies may help them get more features for customer segmentation that they can use for suggestions or promotions.
				\item \textbf{Fraud detection}: The provider may need more information about their customers for detecting fraud. Providers typically create detailed customer profiles to predict their 
				behaviors and detect anomalies. Having demographic and other attributes such as interests and hobbies helps building more accurate customer behavior profiles. Most e-commerce providers 
				are typically under a lot of pressure to detect fraudulent activities on their site as early as possible, so that they can limit their exposure to lawsuits, compliance laws or even loss of 
				reputation.
		\end{enumerate}
		Current approaches to addressing this challenge largely revolve around subscribing to data sources like Experian. This approach has the following shortcomings:
		\begin{enumerate}
			\item The enrichment task is restricted to the attributes provided by the one or two data sources that they buy from. If they need some other attributes about the customers, it is hard to get them.
			\item The selected data sources may have high quality information about attributes, but poor quality about some others. Even if the e-commerce provider knows about other sources, which have those
				attributes, it is hard to manually integrate more sources.
			\item There is no good way to monitor if there is any degradation in the quality of data sources.
		\end{enumerate}
Using the enrichment framework in this context would allow the e-commerce provider to dynamically select the best set of sources for a particular attribute in  particular data enrichment task. 
The proposed framework can switch sources across customer records, if the most preferred source does not have information about some attributes for a particular record. For low confidence values, the 
proposed system uses reconciliation across sources to increase the confidence of the value. The enrichment framework can also continuously monitor and downgrade sources, if there is any loss of quality.


\documentclass[letterpaper]{article}
\usepackage{aaai}
\usepackage{times}
\usepackage{helvet}
\usepackage{courier}
\frenchspacing
\pdfinfo{
/Title (Formatting Instructions for Authors Using LaTeX)
/Subject (AAAI Publications)
/Author (AAAI Press)}
\setcounter{secnumdepth}{0}  
 \begin{document}
% The file aaai.sty is the style file for AAAI Press 
% proceedings, working notes, and technical reports.
%
\title{Data Enrichment Using Web APIs}
\author{Karthik Gomadam, Kunal Verma, Peter .Z.Yeh\\
Accenture Technology Labs
San Jose, CA\\
}
\maketitle
\begin{abstract}
\begin{quote}
AAAI creates proceedings, working notes, and technical reports directly from electronic source furnished by the authors. To ensure that all papers in the publication have a uniform appearance, authors must adhere to the following 
instructions. 
\end{quote}
\end{abstract}
\section{Problem Definition: Data Enrichment}

We define data enrichment as the task of organically growing a data object using information pertinent the object 
across heterogeneous sources, such as enterprise databases, structured and unstructured resources on the Web, and 
file systems. 


\section{Definitions and Functions}

\subsection{Data Object Definition}

A data object $D$ to be enriched consists of two elements $D_k$ and $D_u$. $D_k$ consists of attributes whose 
values are known, which we define formally as $D_k= \lbrace <a,v(a),k_a,k_{v(a)}> ... \rbrace$, where $a$ is 
an attribute, $v(a)$ is the value of the attribute $a$, $k_a$ is the importance of $a$ to $D_k$, and $k_{v(a)}$ 
is the confidence in the correctness of $v(a)$. $D_u$ consists of attributes whose values are unknown and hence 
the targets for enrichment. We define $D_u$ formally as $D_u= \lbrace <a,k_a> ... \rbrace$.
    
\subsection{Attribute Relevance} 

Attribute relevance is a measure of the importance of an attribute to a data object. This relevance is used 
to guide the selection of appropriate data sources for enrichment (see next subsection).

Our definition of importance is based on two intuitions: 1) an attribute has high importance (and hence 
relevance) to a data object if it is uniquely associated with the data object and 2) an attribute has high 
importance if it is highly discriminative w.r.t. the instances of the data object. For example, the attribute 
{\it e-mail contact} should have high importance (and hence relevance) to the {\it Customer} data object 
because it satisfies the two intuitions above. However, the attribute {\it Zip} should have lower relevance
to the {\it Customer} object because it does NOT satisfy the second intuition -- i.e. many customers map
to the same zipcode.

The Data Enrichment Engine (i.e. DEE) captures the above intuitions formally with the following equation:
\begin{equation}
k_a= \frac{1}{1+ e^{-(H_{T}(a) +1)(P'_a - P_a)}}
\end{equation}
where, 
\begin{equation}
H_{T}(a)= - \displaystyle\sum\limits_{v \in a} P_v log P_v
\end{equation}

$P'_a$ is the relative frequency of classes that do not have the attribute $a$ (in the set of all classes 
defined in the system) and $P_a$ is the relative frequency of classes that have $a$. $H_{T}(a)$ is the entropy 
of the past $T$ values of $a$, and serves as a proxy for the uniqueness of the values of $a$ (and hence how 
discriminative is $a$). Because $H_{T}(a)$ is recompute for every $T$ values, relevance of $a$ is also adapted
over time.


\subsection{Data Source Selection} 

One of the primary features of DEE is the ability to automatically select the next data source(s) for 
enrichment, given a data object. 

The selection of the best source to use next must consider two important factors: 1) whether the source will 
be able to provide values if called, and 2) whether the source targets unknown attributes in $D_u$ with high 
relevance. DEE satisfies the first factor by measuring how well known values of $D$ match the inputs required 
by the source. If there is a good match, then the source is more likely to return values when it's called. DEE
also considers the number of times the source has been polled before to prevent ``starvation" of other sources. 

DEE satisfies the second factor by measuring how many high-relevance, unknown attributes the source claims 
to provide. If a source claims to provide a large number of high-relevance, unknown attributes, then the 
system should select the source over others. The second factor serves as the selection bias. 

DEE formally captures these two considerations with the following equation:
\begin{equation}
    F_s = \frac{1}{2^{M-1}}B_s \frac{\displaystyle\sum\limits_{a \in D_k \cap I_s} k_{v(a)}}{|I_s|} 
    	+ \frac{\displaystyle\sum\limits_{a \in D_u \cap O_s} k_{a}}{|D_u|}
\end{equation}
where $B_s$ is the base fitness score of a data source $s$ being considered (this value is randomly set between 
0.5 and 0.75 when the system is initialized), $I_s$ is the set of input attributes to the data source, $O_s$ is 
the set of output attributes from the data source, and $M$ is the number of times the data source has been 
selected in the context of enriching the current data object.

The data source with the highest score $F_s$ that also exceeds a predefined minimum threshold $R$ is selected
as the next source to use for enrichment. The selection (and hence enrichment) process will continue until
either $D_u$ is empty or there are no sources whose score $F_s$ exceeds $R$.


\subsection{Output Value Confidence} 

DEE also computes the confidence in the output value given by a data source for an unknown attribute. 
This confidence is determined using the following formula: 

\begin{equation}
 \label{_output_confidence}
k_{v(a')} = \left\lbrace 
		\begin{array}{ll}
			e^{\lambda(k_{v(a')} - 1)} 						& \mbox{, if } k_{v(a')} \neq \emptyset \\
			e^{ \left( \frac{1}{|V_{a'}|} - 1 \right) } W \nonumber  	& \mbox{, if } k_{v(a')} = \emptyset \\
		\end{array}
		\right.
\end{equation}
where,  
\begin{equation}
	W = \frac{\displaystyle\sum\limits_{a \in D_k \cap I_s} k_{v(a)} }{|I_s|}
\end{equation}
and $V_{a'}$ is the set of output values returned by a data source for an unknown attribute $a'$. If 
multiple values are returned, then there is ambiguity and hence the confidence in the output should 
be discounted. This notion is also captured in the above equation.

The above formula also considers if an output value is corroborated by output values given by previously
selected data sources. If so, then the confidence should be further increased. The $\lambda$ factor is
the corroboration factor ($< 1.0$), and defaults to 1.0.


\subsection{Source Utility and Adaptation}

Once a data source has been called, DEE determines the utility of the source in enriching the data
object of interest. Intuitively, DEE models the utility of a data source as a ``contract" -- i.e. if 
DEE provides a data source with high confidence input values, then it's reasonable to expect the data
source to provide values for all the output attributes that it claims to target. Moreover, these values 
should not be generic and should have low ambiguity. If these expectations are violated, then the data 
source should be penalized heavily. 

On the other hand, if DEE did NOT provide a data source with good inputs, then the source should
be penalized minimally (if at all) if it fails to provide any useful outputs.

DEE captures this notion formally with the following equation:
%\begin{eqnarray}
%U_s &=& W \left( \frac{1}{|O_s|} \left( \displaystyle\sum\limits_{a \in O_s^+} e^{\frac{1}{|V_a|} - 1}k_a^{P_Tv(a)} - \displaystyle\sum\limits_{a \in O_s^-}k_a \right) \right) \nonumber \mbox{ where,}\\
%P_T(v(a)) &=& \left\lbrace \begin{array}{ll} P_T (v(a)) & \mbox{, if } |V_a|= 1 \\ \argmin\limits_{v(a) \in V_a} P_T(v(a)) & \mbox{, if } |V_a| > 1 \end{array} \right.
%\end{eqnarray}
where $O_s^+$ are the output attributes from a data source for which values were returned, $O_s^-$ 
are the output attributes from the same data source for which values were not returned, and $P_T(v(a))$
is the relative frequency of a value $v(a)$ for an attribute $a$ over the past $T$ values returned 
by the data source. 

The utility of a data source $U_s$ from the past $n$ calls are then used to adjust the base fitness score 
of the data source. This adjustment is captured with the following equation
\begin{equation}
 B_s= B_s + \gamma \frac{1}{n} \displaystyle\sum\limits_{1}^{n}U_s(T - i)
\end{equation}
where $B_s$ is the base fitness score of a data source $s$, $U_s(T-i)$ is the utility of the data source $i$ time steps
back, and $\gamma$ is the adjustment rate.


\section{Resolving Ambiguity}

Ambiguity occurs, during the enrichment process, when a data source returns multiple values for an unknown 
attribute. For example, given the following {\it Customer} data object:
\begin{equation}
	(Name: John Smith, City: San Jose, Occupation: NULL)
\end{equation}
a data source may return multiple values for the unknown attribute of {\it Occupation} (e.g. Programmer, 
Artist, etc).

To resolve this ambiguity, DEE will branch the original object -- one branch for each returned value -- 
and each branched object will be subsequently enriched using the same algorithm described above. Hence, 
a single data object may result in multiple objects at the end of the enrichment process. 

DEE then determines the fitness for each resulting object using the following equation:
\begin{equation}
	\frac{\displaystyle\sum\limits_{a \in D_k \cap D_U} k_{v(a)} k_a }{|D_k \cup D_u|}
\end{equation}
The top M objects according to this fitness are then returned to the user.



\end{document}

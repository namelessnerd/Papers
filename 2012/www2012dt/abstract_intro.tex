%\begin{abstract}
%As businesses seek to monetize their data, they are leveraging Web-based delivery mechanisms to provide publicly available data sources. Also, as analytics becomes a central part of many business functions such as customer segmentation, competitive intelligence and fraud detection, many businesses are seeking to enrich their internal data records with data from these data sources. As the number of sources with varying degrees of accuracy and quality proliferate, it is a non-trivial task to effectively select which sources to use for a particular enrichment task. The old model of statically buying data from one or two providers becomes inefficient because of the rapid growth of new forms of useful data such as social media and the lack of dynamism to plug sources in and out.  This challenge is further accentuated by the need to integrate external data with internal databases. 

%In this paper we present the data enrichment framework, a tool that uses data mining and other semantic techniques to automatically guide the selection of sources. The focus of this paper is in outlining the system design and 
%the development methodology, highlighting our experience in using Web application frameworks within the enterprise context. 

%\end{abstract}

\section{Introduction}

As enterprises become more data and analytics driven, many businesses are seeking to enrich their internal data records with data from data sources available on the Web. Consider the example, where the consumer database of a company might have the name and address of its consumers.  Being able to use publicly available data sources such as LinkedIn, White Pages and Facebook to find information such as employment details and interests, can help the company collect additional features for tasks such as customer segmentation. Currently, this is done in a static fashion where a business buys data from one or two sources and statically integrates with their internal data. However, this approach has the following shortcomings:
\begin{enumerate}
  \item \textbf{Inconsistencies in input data}: It is not uncommon to have different set of missing attributes across records in the input data. For example, for some consumers, the street address information might be missing and for others, information about the city might be missing. To address this issue, one must be able to select the data sources at a granular level, based on the input data that is available and the data sources that can be used.
   \item \textbf{Quality of a data source may vary, depending on the input data}: Calling a data source with some missing values (even though they may not be mandatory), can result in poor quality results. In such a scenario, one must be able to find the missing values, before calling the source or use an alternative.
\end{enumerate}
In addition to these, current approaches also suffer from software limitations. These include
\begin{enumerate}
	\item A conventional client server approach to software development. Very few of the enterprise data tools are available as Web based solutions, necessitating complex installation and maintenance overheads. Further, in the 
scenario of using Web based data, where data providers give access keys and restrict data limits, the current approach also adds complexity in key management and ensuring that data limits are not exceeded. 
	\item Emphasis on coding as opposed to configuration for customization: Tradiational systems often require additional implementation for customization. In environments such as ours (a consulting firm), this often leads to 
significant additional man hours to stand up, deploy, and maintain the system. 
	\item Library driven approach: Often times, developer write custom libraries to access data sources (such as LinkedIn, Twitter) and these are used as binary libraries (JARs and DLLs). Given the hetereogenous platforms and 
architectures that are used in an enterprise, reuse of these libraries poses a significant challenge. 
\end{enumerate}


Add one line about how we are addressing. Shorten the shortcomings. :w





In this paper, we present an technical overview of the design and development methodology of the data enrichment framework which attempts to automate many sub-tasks of data enrichment\footnote{We request the reader to visit http://, for a detailed technical report on the enrichment algorithm}. In this paper we will articulate our design methodology and discuss how our framework helps overcome the above mentioned limitations. We leverage existing Web 
development standards such as REST (for communication) and Comet (for streaming updates), existing frameworks in Django (for Web application development) and Celery (for task distribution). 
